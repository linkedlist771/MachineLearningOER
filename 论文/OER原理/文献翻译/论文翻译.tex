\documentclass[a4paper]{article}
\usepackage[margin=1in]{geometry} % 设置边距,符合Word设定
\usepackage{ctex}
\usepackage{lipsum}
\title{\heiti\zihao{2} This is a test for vscode}
\author{\songti Ali-loner}
\date{2020.08.02}
\begin{document}
    \maketitle
\begin{abstract}
    \lipsum[2]
\end{abstract}
\tableofcontents
尽管高熵纳米颗粒是最近才开发出来的,但在一系列新兴的与能源相关的过程和应用中,尤其是在催化方面(图1C)(27-36),已经显示出了巨大的前景。高熵纳米颗粒组成的灵活性使催化活性能够微调,而高熵固溶混合可能提供结构稳定性1,这对苛刻条件下的操作至关重要。例如,非贵金属(CoxMo0.7-x)Fe0.1Ni0.1Cu0.1纳米颗粒已被证明可以克服Co-Mo的不混相性,允许对Co-Mo比和相关的表面吸附性能进行稳健调整。因此,(Co0.25Mo0.45)Fe0.1Ni0.1Cu0.1纳米颗粒在氨分解方面比贵金属钌有四倍的改善,并且在500°C下稳定50小时没有明显降解(36)。在另一个例子中,Pt18Ni26Fe15Co14Cu27纳米颗粒被开发用于电化学析氢,与商用Pt-C催化剂相比,Pt18Ni26Fe15Co14Cu27纳米颗粒表现出较低的起始电位(11 vs . 84 mV),较高的活性(10.98 vs . 0.83 a /mgPt)和优异的稳定性(11)。这些例子体现了高熵纳米颗粒作为高效、经济的催化剂的强大潜力(9,11,12,20,36 - 39)。

与成分相对简单(1 ~ 3个元素)的材料相比,高熵纳米粒子具有两个明显的特点:(1)多元素组合产生的巨大成分空间和(2)随机多元素混合产生的复杂原子构型。前者为催化剂的设计和开发提供了巨大的成分选择,后者使这些材料与传统催化剂有本质区别,因为它们具有不同的吸附位点范围和接近连续的结合能分布模式(40,41)。这些特性是串联反应,涉及许多中间步骤,需要多种功能(28,38,40,42 - 44)。



然而,随着这些机遇,大量可能的成分和复杂的原子排列给这些独特的纳米材料的设计、合成、表征和应用带来了巨大的挑战。首先,考虑到不同组成元素之间的物理化学性质(如原子大小和电子结构)的广泛跨度,以高度可控的方式合成高熵纳米颗粒是困难的。此外,由于相似电子对比的复杂原子构型和多个元素,表征高熵纳米粒子的详细结构,如反应表面和缺陷,是具有挑战性的或仍然缺乏的。
此外,我们对元素组成和合成方法如何影响高熵纳米粒子的结构和性能的知识非常有限。尽管识别这种复杂材料的这些关系是一项艰巨的任务,但理解它们对指导材料设计和优化是至关重要的。


为了应对这一领域日益增长的兴趣、快速发展和巨大挑战,我们旨在突出关于高熵纳米颗粒的合成、结构、表征和应用的重要进展和关键未知因素。我们还讨论了计算引导和数据驱动的潜力和实现
加速对高熵纳米粒子的探索,以及该领域的剩余挑战和未来方向。我们打算鼓励多学科的持续和综合努力,以研究高熵纳米粒子,并探索多维空间中的合成-结构-性质关系。请注意,我们使用术语“高熵纳米颗粒”指的是具有复杂成分(五种或五种以上元素)和固溶结构的粒子,而不是传统的基于某种主观阈值的定义,即金属的每个原子或陶瓷的每个阳离子为1.5 kB,其中kB为玻尔兹曼常数(3,45)。在本文中,虽然我们关注的是高熵纳米粒子,但其基本概念也有望应用于其他纳米材料。
我们预计,随着这些进展,高熵纳米颗粒将在许多领域产生重大影响,特别是催化,这种新材料有可能取代长期存在的贵金属对应物



\section*{高熵纳米颗粒合成}

从热力学的角度来看,高熵纳米粒子的形成是焓和熵(DG = DH - T•DS)竞争的结果。高熵纳米粒子的构型熵随着元素数量的增加而增加,并成为单相混合的驱动力(图2A)。多元素相互作用的焓(DHij)很大程度上取决于组成元素的性质,这直接影响在接近平衡条件下产生的相(图2B)。例如,DHij值高的元素组合(即斥力)会导致不混相和相偏析,而DHij值高的元素组合(即吸引力)则会促进结构有序,如金属间化合物的形成。如果多元素组成中的所有DHij对都接近零值,表明这些元素之间几乎没有相互吸引或排斥,那么熵项就占主导地位,并促进均匀随机元素混合和高熵的形成(图2B)。然而,由于不同元素之间存在较大的物理化学差异(即DHij值的广泛范围),自然单相混合通常是具有挑战性的和罕见的(46,47),当使用近平衡方法(如湿化学)合成多元素纳米颗粒时,相分离结构更为典型(7,48,49)。


马里兰大学(8,50 - 52)的Hu小组发明了一种高温“热冲击”工艺,实现了合成具有广泛成分范围(包括许多不混溶组合)和大元素数(高达8个)的高熵合金(HEA)纳米粒子的初步突破。c)。
这种合成方法的冷却速率是一个重要参数,因为它影响组成元素可以达到的非平衡和结构有序的程度,正如在物理冶金和聚合物固化中使用的众所周知的温度-时间转换图中所描述的(图2D)(8,53,54)。生成的结构可以包括金属玻璃纳米颗粒(在无序晶格中随机混合)、规则HEA纳米颗粒(在晶体晶格中随机混合)、金属间化合物纳米颗粒(亚晶格之间的化学顺序,但每个亚晶格内随机混合)和非均质纳米颗粒(相分离)(8,17)。
此外,热冲击合成的短时间和快速淬火也有助于形成小而均匀的颗粒(8,55),这些颗粒可通过缺陷工程和适当的衬底进一步调制(56-58)。
类似于这种以“震动”为基础的概念,多种多样
其他方法也已经开发出来,这些方法实现了广泛的高熵纳米颗粒,包括气相火花放电(13)、快速辐射加热或退火(12,59,60)、急性化学还原(33,43)、低温氢溢出(61)、溅射(9,62 - 64)、瞬态电合成(15)和等离子体、激光和微波加热(65-67),所有这些方法都具有强烈的动力学驱动过程。这些快速的、冲击型合成也足以使纳米催化剂的高效制造成为可能(10,12,13,68,69)。



Ellingham图可以用来指导高熵纳米颗粒的热化学合成,通过说明组成元素的氧化势作为温度的函数(图2E)。尽管最初是为大宗冶金开发的,但我们发现Ellingham图也适用于纳米尺度的冲击型反应(14,20)。一般来说,靠近埃林厄姆图顶部的元素,如贵金属和Fe、Co、Cu,其氧化电位比l (i)小。e。、r e m o r e e s l y r e d u c e d) n d可以形成合金纳米粒子通过高温合成,如八的头脑PtPdFeCoNiAuCuSn纳米颗粒(图2)(8)。相比之下,图的底部附近的元素,如锆、钛、高频、Nb,有更大的氧化电位,可以形成熵氧化物纳米颗粒,如(ZrCeHfCaMgTiLaYGdMn)牛(图2 g)(20)。
对于中间的元素,如Mo, W, a n d mn (as s h o W n n g r en n n F i g)。在中等氧化势下,人们探索了不同的合成策略,可以在元素的金属态和氧化态之间切换,从而扩大了可能性
高熵合金或氧化物元素空间(14,20,70)。除了高熵的氧化物(8),其他的高熵化合物(例如硫化物和碳化物)也被合成,其大小、形状和相的范围很广(8,12,14 - 18,26 - 30,71)。

\section*{先进表征}

高熵纳米颗粒应表现为单相结构,表现为组成元素的均匀和随机混合。然而,描述这种多元素的随机混合及其协同作用是非常具有挑战性的。传统的技术,如粉末x射线衍射(XRD, l = 1.5418 Å)、扫描和透射电子显微镜(SEM和TEM分别)和x射线光电子能谱(XPS),可以帮助确定基本的相结构、形貌、元素分布和价态,但可能缺乏所需的分辨率来解耦多元素混合。基于同步x射线的技术使用了更短的波长(例如,l = 0.2113 Å),可以提供高分辨率来更好地理解高熵纳米粒子的原子排列、键合和配位以及电子特性(图3A)。例如,同步射线衍射可以检测整体的相结构
以及可能的不混相和杂质在高熵纳米颗粒更精确(49,72),确认是否实现了高熵混合。x射线吸收光谱(XAS)是一种元素特异性技术,可用于研究每种元素的原子和/或局部配位环境,这对于理解高熵纳米粒子中多元素混合和可能的短程或局部有序至关重要(14,73,74)。最后,硬x射线光电子能谱(HAXPES)可以揭示高熵纳米粒子的电子结构(如价带和d带中心),这与关键反应中间体的吸附和结合能密切相关,有助于合理解释相应的催化活性(75)

虽然x射线技术可以提供统计分析,但基于电子显微镜的技术对于直接可视化颗粒的大小和分布、相、结构、成分和化学环境是至关重要的。例如,原位透射电镜已被用于研究高温激波法合成的纳米颗粒,揭示了纳米颗粒的形成过程及其在缺陷碳基质上的分散和稳定性(76)。另一种可以满足更高通量和更高分辨率需求的先进方法是四维扫描透射电子显微镜(4D-STEM)(图3B)(14,77)。4D-STEM使用一个小探针(~1 nm)扫描一个大的几何区域,面积可达~1 × 1 mm2,因此能够快速和高分辨率地描述高熵纳米颗粒的局部晶格畸变、结构不均一性和近程有序(78)。A s s h o w n n F i g。在高熵纳米粒子上可以得到3 B, local diff ra ct on pa tterns,通过比较局部相结构与平均相结构的差异(14,77),可以得到纳米粒子内部相应的应变图,表明潜在晶格畸变和应变


要对三维原子结构进行更高级的表征,原子电子断层扫描有p h y (a E T)有s p ro v b E E h h ho d选项(图3C)(79-81)。最近,AET技术取得了很大进展,能够解析含有8个元素的高熵金属玻璃纳米粒子的三维原子结构:C o, N i, R u, R h, P d, a g, ir, a N d P t (F ig)。c)(16)。
B e c u s e t h e i m g e c o n t r s t o f e t d e p e n d s o n原子序数,让目前仅足够敏感分类的八个元素n t o t h r e e t y p e s: c o n d n i s t y p e 1 (g r e e n);Ru, Rh, Pd, Ag为2型(蓝色);红外和Pt 3型(红色)(图3,D和E)。图3 D显示的3 D原子结构highentropy金属玻璃纳米颗粒,在t h E t y p E 1, 2, 3 n D t o m s r E u n i f o r m l y D i s致敬。三维原子结构揭示了四种不同的晶体状中等排列顺序,包括面心立方、六边形密排、体心立方和共存于高熵纳米粒子中的简单立方结构(图3E)。这些结果提供了直接的实验证据,支持金属玻璃的高效团簇填充模型的一般框架(82),并展示了AET技术如何使研究人员能够在单原子水平上研究高熵纳米颗粒的三维结构。


\section*{多功能催化活性}
此前,高熵材料,特别是高熵合金,主要用于结构工程应用(3)。
Wang等人首次证明了高熵合金纳米颗粒可以作为热催化的高效催化剂(8,36)。在催化过程中,反应物或中间体与催化剂表面的结合不能太强也不能太弱(根据Sabatier原理),这样才能使性能最大化,因此活度对结合能的依赖表现出“火山图”(83,84)。作为一种方案,它是可以用的。个别元素(如Co、Mo、Fe、Ni、Cu)的A、b、n、n和n的分布规律由于其相对固定的结构和吸附位点,往往呈现出明显的峰值。然而,当多种元素混合成高熵合金(如CoMoFeNiCu)时,它们的吸附能可以通过电子杂化转化为一个加宽的、多峰的、几乎连续的光谱。最近,Löffler等。
报道了在高熵催化剂上电催化的“电流波”模式,其中观察到多个拐点和电流平台,这强烈表明在高熵纳米颗粒中存在多个活性位点中心(40,42)。



由于独特的结合能分布,高熵纳米颗粒可以很容易地调整,以获得最佳催化性能所需的表面性能(28,40,62)。例如,在NH3分解反应(2NH3→N2 +3H2)中,通过理论分析,从理论上提出非贵金属Co-Mo合金由于对*N的吸附优化而优于Ru(图4B中的火山图)(85);然而,这种设计受到Co和Mo的不混相性的阻碍。最近,使用热冲击法合成的(CoxMo1-x)70(FeNiCu)30 HEA纳米颗粒证明了合金钴基催化剂(36)。
在给定的反应条件下,可以通过调节Co:Mo元素比来优化氮吸附能(DEN),从而达到优于活性最高的单金属催化剂Ru的性能(图4C)。
在许多其他体系(9,15,20,43,62,73,75,86,87)中也观察到类似的高熵纳米催化剂的高性能,证明了多元素设计和成分可调的重要性。值得注意的是,活性位点的多样性和异质性会导致局部活性(<50 nm)的统计变化,但总体可重复性能(88)。



理论上,火山图可以解释为第一性原理计算研究中线性尺度关系(LSR)的结果(85,89)。LSR认为,在复杂或多步反应中,反应中间体(如O*或OH*)的吸附能线性关联或线性缩放(83);换句话说,反应物的强吸附很可能导致生成物的强吸附(即难以解吸),从而大大减慢反应速度(90)。人们提出了许多策略来规避纳米颗粒催化剂设计中的LSR,包括引入共吸附剂和系链剂、启动子、配体和组成元素之间具有复杂协同作用的新合金(83,85)。与简单的催化剂相比,高熵纳米颗粒具有复杂的原子构型、不同的吸附位点和可调的结合能,这可能带来一系列新的机会(83)。例如,Wu等人报道了用于析氢反应(2H2O→O2+2H2)的高贵IrPdPtRhRu HEA纳米颗粒,并发现该材料与单个金属(Ir、Pd、Pt、Rh和Ru)相比表现出优越的性能(图4D)(75)。更重要的是,IrPdPtRhRu的转换频率远远超出了传统LSR理论的预期(图4E中的蓝色区域),表明HEA有能力绕过LSR预测。



此外,高熵纳米颗粒的宽频吸附能景观尤其适用于催化串联和复杂反应,通常需要不同的活性位点和多个反应中间体的吸附,以达到整体的高活性和/或选择性(27,71)。例如,在涉及复杂的12电子转移和一系列中间体的乙醇氧化反应中,高熵的PtPdRuRhOsIr (PGM-HEA)纳米颗粒不仅表现出比单金属催化剂及其物理混合物更高的活性,而且能够实现更高的12电子选择性以完成氧化生成CO2(图4F)(43,62,71)。
在另一个例子中,Ru22Fe20Co18Ni21Cu19 HEA纳米颗粒在氮还原反应中表现出了较高的活性和选择性(NRR: N2 + 3h2→2NH3)(38)。理论分析发现HEA中的Fe适合N2的吸附和解离,而附近的Co-Cu和Ru-Ni组合则有利于H2的吸附和解离,说明多功能活性位点对整体高效合成NH3的重要性。
同样,高性能的高熵纳米颗粒也被报道用于其他配合物和多步反应,如二氧化碳减少反应(39,71,91),与各种化学物质如甲醇(11,13,43)


\section*{稳定性}

高熵纳米颗粒具有增强催化应用稳定性的潜力,与体积尺寸的纳米颗粒具有改善结构稳定性的特点(3,45,53,92)相似。热力学上,高熵的性质有利于高熵纳米颗粒(DG = DH - TDS)的形成和稳定,特别是在高温下,TDS项更为明显(20,73,93)。原位透射电镜分析揭示了高熵合金和氧化物纳米颗粒的稳定性,即使在高达1073 K的温度下,其尺寸分布、粒子分散和固溶相也保持不变(13,20,73)。在动力学上,由于不同元素的尺寸不匹配和由此产生的晶格畸变,高熵混合也可能提高结构稳定性,这可能导致较大的扩散势垒,有助于防止相偏析,特别是在低温(2,20,53,70)下。作为一个例子,Ru原子在RuRhCoNiIr HEA纳米粒子中的扩散系数被模拟为比扩散系数低两个数量级
Ru在双金属Ru - ni中的活性,表明在HEA纳米颗粒中具有更好的动力学稳定性(73)。
影响催化剂稳定性的另一个重要因素是催化剂和载体之间的界面结合,以避免颗粒聚集。高温激波合成可以使高熵纳米粒子与衬底之间的界面稳定性更好(76,87,94)。实验表明,高熵催化剂在高温和电化学催化反应中的稳定性能(11 - 13,15,20,36 - 38,43,44,73,75)证明了其稳定性。
然而,熵稳定作用可能会受到限制,在恶劣的条件下容易发生表面重建(95-97)。例如,Shahbazian-Yassar等人研究了Fe0.28Co0.21Ni0.20Cu0.08Pt0.23 HEA纳米颗粒的原位氧化,并观察到非贵金属元素的表面氧化,而HEA纳米颗粒的核心在富含pt的成分下保持稳定(图4G)。定性分析表明,HEA纳米颗粒表现出对数氧化动力学,在400°C的氧化环境中暴露40分钟后,HEA核和氧化壳结构稳定(图4H)。相比之下,纯Co纳米颗粒发生突变氧化动力学,在~1 min内氧化。高熵催化剂的表面重建或转化往往在电化学反应中更为明显,其熵稳定作用不如化学浸出和电化学氧化还原作用深刻。尽管如此,许多研究已经报道了高熵纳米颗粒在不同电化学条件下的稳定性能,特别是与较少元素的同类相比(12,29,75,86,87)。


\section*{高通量筛选}
尽管在一些情况下观察到优越的催化性能,但仍然不知道如何开发用于靶向催化反应方案的高熵纳米颗粒。
此外,由于复杂的微观结构和结合能分布模式(40,42,98),在高熵纳米颗粒中识别催化活性位点具有挑战性。
这些问题可以通过利用新兴的高通量(64,99,100)和数据驱动的材料发现方法(27,71,101 - 103)来解决。



在计算上,基于第一性原理的方法已经被开发用来预测高熵纳米颗粒的组成、结构和性质关系(84,100,104)。此外,通过遵循来自高熵材料的经验规则(46,73)或使用相图计算(CALPHAD)方法(大大减少了参数空间(105),证明了高通量计算可用于多元素成分的相预测(图5A)。这两种方法都能够筛选数百万种元素成分(图5A)。
然而,这些计算大多基于大块材料的热力学平衡考虑,由于其体积小,且在非平衡条件下合成,可能不容易转移到高熵纳米颗粒


对于预测高熵纳米颗粒的功能特性(如催化),还存在额外的挑战,如建立精确的原子填充模型和确定结合位点(11)。最近,rosmeisl等人开发了一种结合监督学习的高通量计算方法,用于探索高熵纳米颗粒中的随机原子构型,并预测其催化吸附能(图5B)(27,71,101)。作者还模拟了高熵催化剂的近连续结合能分布模式(图5B)。在这些计算的基础上,实验实现了用于氧还原和CO2还原的高性能多元素催化剂(27,71,101,106)。其他基于机器学习(ML)的方法正在开发,以有效地探索吸附在多元素表面上的构型,包括可变吸附剂覆盖范围、多种吸附种类和表面重建对催化性能的影响(107)。



在实验上,研究人员已经证明了多元素催化剂的组合合成和高通量筛选(64,108 - 110)。例如,Ludwig等人利用多种金属源的共溅射,在薄膜衬底上实现了数百种高熵成分(每批约342种)的组合合成,以及高通量表征,包括能量弥散光谱(成分)、XRD(结构)和扫描液滴池(电化学),以快速筛选这些2D薄膜样品,以快速发现催化剂(64,111 - 113)。
高熵纳米颗粒的直接高通量合成和筛选也已经实现(9,72)。通过组合共溅射到离子液体中(每腔约40毫升,共64腔),Ludwig等人演示了基于crmnfeconi的HEA纳米颗粒的合成,固定在不同成分的微电极上,这导致了对氧还原反应具有特殊活性的Cr9Mn60Fe9Co11Ni11的发现(9)。
报道了具有从二元到八元PtPdRhRuIrNiCoFe不同元素组合的超细均匀HEA纳米颗粒的高通量合成(图5C)(72)。在此过程中,采用油墨印刷不同的金属前驱体溶液,然后进行高温辐射冲击合成,以获得不同成分的均匀微观结构。扫描液滴细胞筛选之后,发现了用于氧还原反应的高性能PtPdFeCoNi HEA催化剂,其催化性能通过常规旋转圆盘电极测量得到验证(72)。因此,组合合成和高通量筛选管道为加速探索高熵纳米粒子提供了一个新的范式。


\section*{基于机器学习加速的主动探索}
ML是一种极好的工具,可以通过广泛预测未测成分(ML中的泛化过程),引导探索快速找到性能最佳(ML中的主动学习),定量理解成分和过程-结构-性质关系(ML中的特征分析)来加速材料发现(27,63,71,101 - 103,114,115)。作为一个例子,ML预测已经被用于指导三元中熵PtFeCu催化剂的设计,通过(i)模型的建立和仿真数据的生成,(ii) ML和模拟数据的拟合,(iii) ML在更大的组成空间中广泛的探索和筛选,(iv)实验验证和反馈之前的仿真和ML模型(图5D)(116)来说明闭环过程。在Ag-Ir-Pd-Pt-Ru空间的多元素催化剂中,通过将计算预测与ML结合,使用基于薄膜的高通量合成和筛选进行数据反馈和模型细化,也证明了类似的过程,从而形成闭环优化协议,提高对高性能催化剂的预测能力(63)。


尽管取得了这些进展,但目前的努力最多只能覆盖不到1\%的高熵纳米粒子(113)中可用成分。因此,指导优化和仔细采样对于识别重要数据点至关重要,从而节省勘探工作。这可以通过使用主动学习方法(如贝叶斯优化和强化学习)来实现(103,117 - 119)。例如,rosmeisl等人。
利用贝叶斯优化和高斯过程代理函数模型,基于计算数据发现多元素催化剂(120)。基于贝叶斯优化的约150次迭代,发现了目标属性的许多重要局部最优值,说明了主动学习在探索广阔的多维空间方面的巨大前景。
这种方法还可以与HEA表面的图网络描述符和神经网络相结合,以加速表面和吸附特性的替代计算模型的发展(107)。
主动学习还可以实现多目标优化,这在高熵纳米颗粒的开发中尚未实现,但非常适合于实现同时具有高活性、高选择性和高稳定性的优良催化剂的目标


对于像高熵纳米粒子这样复杂的材料系统来说,理解合成-结构-性质的关系总是具有挑战性的。一些初步的研究使用理论模型和神经网络将多元素协同解耦为配体效应(即不同的元素)和配位效应(即不同的结构),从而将结构特征与其催化性能联系起来(101)。因此,通过数据训练的数学模型可以逐步学习并促进对高熵纳米颗粒催化剂的性能预测、指导优化和基本理解,而不是建立在对催化机理清晰了解的传统合成-结构-性能关系(图5E)。这种训练过的模型(以高斯过程模型或神经网络的形式)可能成为研究高熵纳米颗粒等复杂材料的新规范



\section*{结论与展望}
高熵纳米粒子已经取得了很大的进展,但要进一步推进它们在许多领域需要继续努力,如合成方法、先进的表征、基本理解以及应用和数据驱动的发现,如下文所述。

可调谐合成是目前高熵纳米粒子探索最多的方面,现在需要精确度。考虑到高熵纳米颗粒中元素差异和成分复杂性造成的不混溶性,合成必须继续依靠温度、力、压力、能量场等方面的非平衡方法,才能达到混合均匀和粒径小的目的。此外,我们还需要学习如何平衡非平衡合成与精密的结构或形态控制在尺寸,相位,形状,方面,和表面装饰,这将需要大量的努力和知识从现有的湿化学



高熵纳米粒子研究目前缺乏的一个重要方面是对高熵纳米粒子的表面、缺陷和元素分布的基本理解,这将对催化性能产生深远影响。
我们还没有建立表面或界面元素分离、重构和电子结构的基本知识,特别是它们在催化操作条件下的动态演化。将最先进的原位电子显微镜,如原位液体和环境显微镜,集成到先进的原子分辨率化学分析和原子结构成像中,将为催化应用的高熵纳米材料的活性位点和反应途径的基本理解提供有价值的见解。此外,我们设想将原子分辨率原位环境显微镜与ml辅助的数据采集和分析相结合,使我们能够捕获催化反应过程中高熵纳米材料的关键动态变化。高熵纳米粒子的表面原子结构、晶格应变、化学扩散和电子结构的演化等信息将被获得,为理论计算和理解反应路径提供可靠的输入。



高熵纳米颗粒在高性能催化方面有很大的前景,尤其适用于需要不同活性位点组合的多步和串联反应。然而,如何正确设计高熵纳米粒子以最适合这些反应方案仍然是一个悬而未决的问题。此外,如何识别活性位点和了解性能起源也不清楚。尽管基于传统途径的催化剂发现是可能的,但高熵纳米颗粒研究将从高通量方法和数据挖掘的进步中获得极大的好处。
目前,组合合成和高通量筛选大多局限于薄膜样品和简单的电化学反应。此外,高通量计算往往以简化或计算效率的精度为代价,导致筛选结果与实际性能趋势之间存在一些差异。
因此,这些数据驱动的方法可能需要在下一阶段的研究中付出最大的努力。


许多已发表的研究结果展示了具有有趣属性的不同组成,但需要更系统和标准化的报告来充分利用这些“昂贵”的数据(102)。
因此,应该为共享数据存储库建立报告标准,以便更好地收集和分析知识。一些这样的努力已经在进行,例如建立材料数据库,用于存档各种材料的三维原子坐标和化学种类,包括AET测定的多元素和高熵纳米颗粒(121)。实验确定的高熵纳米粒子三维原子模型可以与计算和ML方法相结合,在基本水平上了解它们的结构-性质关系。我们希望,随着对高熵纳米颗粒合成-结构-性能关系知识的不断扩展,结合ml引导的优化和筛选的集成材料发现工作流将很快成为可能,以加快这一有前途的领域的进展,特别是同时实现高活性、选择性、稳定性和低成本的多目标优化。

\end{document}